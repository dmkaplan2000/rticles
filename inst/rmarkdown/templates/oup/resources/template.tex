% Colors for colorlinks
$if(colorlinks)$
\PassOptionsToPackage{dvipsnames,svgnames*,x11names*}{xcolor}
$endif$

% Set document class options
\documentclass[webpdf$if(papersize)$,$papersize$$else$,large$endif$$if(document-style)$,$document-style$$else$,contemporary$endif$$if(numbersections)$$else$,unnumsec$endif$$if(namedate)$,namedate$endif$$if(classoptions)$,$for(classoptions)$$classoptions$$sep$,$endfor$$endif$]{oup-authoring-template}

% one column
$if(onecolumn)$\onecolumn$endif$

%\usepackage{showframe}

% line numbers
$if(numberlines)$\usepackage{lineno}\linenumbers$endif$

% Theorem stuff from template
\theoremstyle{thmstyleone}%
\newtheorem{theorem}{Theorem}%  meant for continuous numbers
%%\newtheorem{theorem}{Theorem}[section]% meant for sectionwise numbers
%% optional argument [theorem] produces theorem numbering sequence instead of independent numbers for Proposition
\newtheorem{proposition}[theorem]{Proposition}%
%%\newtheorem{proposition}{Proposition}% to get separate numbers for theorem and proposition etc.
\theoremstyle{thmstyletwo}%
\newtheorem{example}{Example}%
\newtheorem{remark}{Remark}%
\theoremstyle{thmstylethree}%
\newtheorem{definition}{Definition}

% Packages I think are necessary for basic Rmarkdown functionality
\usepackage{hyperref}
\usepackage{graphicx}
\usepackage{listings}
\usepackage{xcolor}
\usepackage{fancyvrb}
\usepackage{framed}

% Link coloring
\hypersetup{breaklinks=true,
            bookmarks=true,
            pdfauthor={$author-meta$},
            pdftitle={$title-meta$}
            $if(colorlinks)$
            ,
            colorlinks=true,
            citecolor=$if(citecolor)$$citecolor$$else$blue$endif$,
            urlcolor=$if(urlcolor)$$urlcolor$$else$blue$endif$,
            linkcolor=$if(linkcolor)$$linkcolor$$else$blue$endif$
            $endif$
}


% For knitr::kable functionality
\usepackage{booktabs}
\usepackage{longtable}

%% To allow better options for figure placement
%\usepackage{float}

% % Packages that are supposedly required by OUP sty file
% \usepackage{amssymb, amsmath, geometry, amsfonts, verbatim, endnotes, setspace}

% For code highlighting I think
\DefineVerbatimEnvironment{Highlighting}{Verbatim}{commandchars=\\\{\}}
\definecolor{shadecolor}{RGB}{248,248,248}
\newenvironment{Shaded}{\begin{snugshade}}{\end{snugshade}}
\newcommand{\AlertTok}[1]{\textcolor[rgb]{0.94,0.16,0.16}{#1}}
\newcommand{\AnnotationTok}[1]{\textcolor[rgb]{0.56,0.35,0.01}{\textbf{\textit{#1}}}}
\newcommand{\AttributeTok}[1]{\textcolor[rgb]{0.77,0.63,0.00}{#1}}
\newcommand{\BaseNTok}[1]{\textcolor[rgb]{0.00,0.00,0.81}{#1}}
\newcommand{\BuiltInTok}[1]{#1}
\newcommand{\CharTok}[1]{\textcolor[rgb]{0.31,0.60,0.02}{#1}}
\newcommand{\CommentTok}[1]{\textcolor[rgb]{0.56,0.35,0.01}{\textit{#1}}}
\newcommand{\CommentVarTok}[1]{\textcolor[rgb]{0.56,0.35,0.01}{\textbf{\textit{#1}}}}
\newcommand{\ConstantTok}[1]{\textcolor[rgb]{0.00,0.00,0.00}{#1}}
\newcommand{\ControlFlowTok}[1]{\textcolor[rgb]{0.13,0.29,0.53}{\textbf{#1}}}
\newcommand{\DataTypeTok}[1]{\textcolor[rgb]{0.13,0.29,0.53}{#1}}
\newcommand{\DecValTok}[1]{\textcolor[rgb]{0.00,0.00,0.81}{#1}}
\newcommand{\DocumentationTok}[1]{\textcolor[rgb]{0.56,0.35,0.01}{\textbf{\textit{#1}}}}
\newcommand{\ErrorTok}[1]{\textcolor[rgb]{0.64,0.00,0.00}{\textbf{#1}}}
\newcommand{\ExtensionTok}[1]{#1}
\newcommand{\FloatTok}[1]{\textcolor[rgb]{0.00,0.00,0.81}{#1}}
\newcommand{\FunctionTok}[1]{\textcolor[rgb]{0.00,0.00,0.00}{#1}}
\newcommand{\ImportTok}[1]{#1}
\newcommand{\InformationTok}[1]{\textcolor[rgb]{0.56,0.35,0.01}{\textbf{\textit{#1}}}}
\newcommand{\KeywordTok}[1]{\textcolor[rgb]{0.13,0.29,0.53}{\textbf{#1}}}
\newcommand{\NormalTok}[1]{#1}
\newcommand{\OperatorTok}[1]{\textcolor[rgb]{0.81,0.36,0.00}{\textbf{#1}}}
\newcommand{\OtherTok}[1]{\textcolor[rgb]{0.56,0.35,0.01}{#1}}
\newcommand{\PreprocessorTok}[1]{\textcolor[rgb]{0.56,0.35,0.01}{\textit{#1}}}
\newcommand{\RegionMarkerTok}[1]{#1}
\newcommand{\SpecialCharTok}[1]{\textcolor[rgb]{0.00,0.00,0.00}{#1}}
\newcommand{\SpecialStringTok}[1]{\textcolor[rgb]{0.31,0.60,0.02}{#1}}
\newcommand{\StringTok}[1]{\textcolor[rgb]{0.31,0.60,0.02}{#1}}
\newcommand{\VariableTok}[1]{\textcolor[rgb]{0.00,0.00,0.00}{#1}}
\newcommand{\VerbatimStringTok}[1]{\textcolor[rgb]{0.31,0.60,0.02}{#1}}
\newcommand{\WarningTok}[1]{\textcolor[rgb]{0.56,0.35,0.01}{\textbf{\textit{#1}}}}

% use upquote if available, for straight quotes in verbatim environments
\IfFileExists{upquote.sty}{\usepackage{upquote}}{}

% For making Rmarkdown lists
\providecommand{\tightlist}{%
  \setlength{\itemsep}{0pt}\setlength{\parskip}{0pt}}

% Counters for addresses and footnotes
\newcounter{correspcnt} % For author footnotes
\renewcommand*{\thecorrespcnt}{\fnsymbol{correspcnt}}
\newcounter{addrcnt} % For author addresses

% Macros for dealing with affiliations, footnotes, etc.
\makeatletter

\def\MyNewLabel#1#2#3{\expandafter\gdef\csname #1@#2\endcsname{#3}}

\def\MyRef#1#2{\@ifundefined{#1@#2}{???}{\csname #1@#2\endcsname}}

\newcommand*\ifcounter[1]{%
  \ifcsname c@#1\endcsname
    \expandafter\@firstoftwo
  \else
    \expandafter\@secondoftwo
  \fi
}

\newcommand*\addrlblbycode[1]{%
  \ifcounter{ADDRLBL@#1}
    {}
    {\refstepcounter{addrcnt}\newcounter{ADDRLBL@#1}\setcounter{ADDRLBL@#1}{\value{addrcnt}}}%
    \arabic{ADDRLBL@#1}%
}

\newcommand*\addrbycode[1]{%
  \ifcounter{ADDR@#1}
    {}
    {\newcounter{ADDR@#1}%
     \address[\addrlblbycode{#1}]{\MyRef{ADDRTXT}{#1}}}%
}

\newcommand*\corresplblbycode[1]{%
  \ifcounter{CORRESPLBL@#1}
    {}
    {\refstepcounter{correspcnt}\newcounter{CORRESPLBL@#1}\setcounter{CORRESPLBL@#1}{\value{correspcnt}}}%
    \fnsymbol{CORRESPLBL@#1}%
}

\newcommand*\correspbycode[1]{%
  \ifcounter{CORRESP@#1}
    {}
    {\newcounter{CORRESP@#1}%
     \corresp[\corresplblbycode{#1}]{\MyRef{CORRESPTXT}{#1}}}%
}

\makeatother


% Create labels for Addresses if the are given in Elsevier format
$for(address)$
 $if(address.code)$
  \MyNewLabel{ADDRTXT}{$address.code$}{%
 $elseif(address.id)$
  \MyNewLabel{ADDRTXT}{$address.id$}{%
 $endif$
 $if(address.address)$$address.address$$else$%
  $if(address.department)$\orgdiv{$address.department$}, $endif$%
  $if(address.organization)$\orgname{$address.organization$}, $endif$%
  $if(address.street)$\orgaddress{%
   $elseif(address.postcode)$\orgaddress{%
   $elseif(address.state)$\orgaddress{%
   $elseif(address.country)$\orgaddress{%
  $endif$%
  $if(address.street)$\street{$address.street$}, $endif$%
  $if(address.postcode)$\orgdiv{$address.postcode$}, $endif$%
  $if(address.state)$\orgdiv{$address.state$}, $endif$%
  $if(address.country)$\orgdiv{$address.country$}$endif$%
  $if(address.street)$}%
   $elseif(address.postcode)$}%
   $elseif(address.state)$}%
   $elseif(address.country)$}%
  $endif$%
 $endif$%
 }
$endfor$

% Create labels for Footnotes if they are given in Elsevier format
$for(footnote)$
$if(footnote.code)$
\MyNewLabel{CORRESPTXT}{$footnote.code$}{$footnote.text$}
$endif$
$endfor$

% Part for setting citation format package: natbib
$if(natbib)$
\bibliographystyle{$if(namedate)$abbrvnat$else$plain$endif$}
$endif$

% Part for indenting CSL refs
% Pandoc citation processing
$if(csl-refs)$
\newlength{\csllabelwidth}
\setlength{\csllabelwidth}{3em}
\newlength{\cslhangindent}
\setlength{\cslhangindent}{1.5em}
% for Pandoc 2.8 to 2.10.1
\newenvironment{cslreferences}%
  {$if(csl-hanging-indent)$\setlength{\parindent}{0pt}%
  \everypar{\setlength{\hangindent}{\cslhangindent}}\ignorespaces$endif$}%
  {\par}
% For Pandoc 2.11+
\newenvironment{CSLReferences}[2] % #1 hanging-ident, #2 entry spacing
 {% don't indent paragraphs
  \setlength{\parindent}{0pt}
  % turn on hanging indent if param 1 is 1
  \ifodd #1 \everypar{\setlength{\hangindent}{\cslhangindent}}\ignorespaces\fi
  % set entry spacing
  \ifnum #2 > 0
  \setlength{\parskip}{#2\baselineskip}
  \fi
 }%
 {}
\usepackage{calc} % for calculating minipage widths
\newcommand{\CSLBlock}[1]{#1\hfill\break}
\newcommand{\CSLLeftMargin}[1]{\parbox[t]{\csllabelwidth}{#1}}
\newcommand{\CSLRightInline}[1]{\parbox[t]{\linewidth - \csllabelwidth}{#1}\break}
\newcommand{\CSLIndent}[1]{\hspace{\cslhangindent}#1}
$endif$

% Pandoc header
$for(header-includes)$
$header-includes$
$endfor$

\begin{document}

\journaltitle{$if(journal)$$journal$$else$Journal Title Here$endif$}
\DOI{DOI HERE}
\copyrightyear{YYYY}
\pubyear{YYYY}
\access{Advance Access Publication Date: Day Month Year}
\appnotes{Paper}

\firstpage{1}

$if(subtitle)$\subtitle{$subtitle$}$endif$

\title[$short_title$]{$title$}

\newcounter{thisauthcorresp} % For storage if author is corresponding author
\newcounter{thisauththanks} % For storage if author has thanks

$for(authors)$

\author[%
$if(authors.address)$\refstepcounter{addrcnt}\arabic{addrcnt}$endif$%
$if(authors.corresponding_author)$$if(authors.address)$,$endif$\refstepcounter{correspcnt}\setcounter{thisauthcorresp}{\value{correspcnt}}\fnsymbol{thisauthcorresp}$endif$%
$if(authors.thanks)$$if(authors.address)$,$elseif(authors.corresponding_author)$,$endif$\refstepcounter{correspcnt}\setcounter{thisauththanks}{\value{correspcnt}}\fnsymbol{thisauththanks}$endif$%
]{$authors.name$}

$if(authors.address)$\address[\arabic{addrcnt}]{$authors.address$}$endif$

$if(authors.corresponding_author)$\corresp[\fnsymbol{thisauthcorresp}]{Corresponding author. \href{mailto:$authors.email$}{$authors.email$}}$endif$
$if(authors.thanks)$\corresp[\fnsymbol{thisauththanks}]{$authors.thanks$}$endif$

$endfor$

$for(author)$

\author[%
$for(author.affiliation)$\addrlblbycode{$author.affiliation$}$sep$,$endfor$%
$if(author.corresponding_author)$$if(author.affiliation)$,$endif$\refstepcounter{correspcnt}\setcounter{thisauthcorresp}{\value{correspcnt}}\fnsymbol{thisauthcorresp}$endif$%
$if(author.thanks)$$if(author.affiliation)$,$elseif(author.corresponding_author)$,$endif$\refstepcounter{correspcnt}\setcounter{thisauththanks}{\value{correspcnt}}\fnsymbol{thisauththanks}$endif$%
$if(author.footnote)$$if(author.affiliation)$,$elseif(author.corresponding_author)$,$elseif(author.thanks)$,$endif$$for(author.footnote)$\corresplblbycode{$author.footnote$}$sep$,$endfor$$endif$%
]{$author.name$}

$for(author.affiliation)$
\addrbycode{$author.affiliation$}
$endfor$

$if(author.corresponding_author)$\corresp[\fnsymbol{thisauthcorresp}]{Corresponding author. \href{mailto:$author.email$}{$author.email$}}$endif$
$if(author.thanks)$\corresp[\fnsymbol{thisauththanks}]{$author.thanks$}$endif$

$for(author.footnote)$
\correspbycode{$author.footnote$}
$endfor$

$endfor$

\authormark{$for(author/first)$$it.name$$endfor$$if(author/rest/rest)$ et al.$else$$for(author/rest)$ \and $author.name$$endfor$$endif$}

\received{Date}{0}{Year}
\revised{Date}{0}{Year}
\accepted{Date}{0}{Year}

%\editor{Associate Editor: Name}

%\abstract{
%\textbf{Motivation:} .\\
%\textbf{Results:} .\\
%\textbf{Availability:} .\\
%\textbf{Contact:} \href{name@bio.com}{name@bio.com}\\
%\textbf{Supplementary information:} Supplementary data are available at \textit{Briefings in Bioinformatics}
%online.}

\abstract{$abstract$}

\keywords{$for(keywords)$$keywords$$sep$; $endfor$}

% \boxedtext{
% \begin{itemize}
% \item Key boxed text here.
% \item Key boxed text here.
% \item Key boxed text here.
% \end{itemize}}

\maketitle

$for(include-before)$
$include-before$
$endfor$

$body$

%%%%%%%%%%%%%%

%\begin{appendices}
%\end{appendices}

$if(competing_interests)$
\section{Competing interests}

$competing_interests$
$endif$

$if(author_contributions)$
\section{Author contributions statement}

$author_contributions$
$endif$

$if(acknowledgments)$
\section{Acknowledgments}

$acknowledgments$
$endif$

$if(natbib)$
$if(bibliography)$

$if(biblio-title)$
$if(book-class)$
\renewcommand\bibname{$biblio-title$}
$else$
\renewcommand\refname{$biblio-title$}
$endif$
$endif$

\bibliography{$for(bibliography)$$bibliography$$sep$,$endfor$}
$endif$
$endif$


%% sample for biography with author's image
%\begin{biography}{{\color{black!20}\rule{77pt}{77pt}}}{\author{Author Name.} This is sample author biography text. The values provided in the optional argument is meant for sample purpose. There is no need to include the width and height of an image in the optional argument for live articles. This is sample author biography text this is sample author biography text this is sample author biography text this is sample author biography text this is sample author biography text this is sample author biography text this is sample author biography text this is sample author biography text.}
%\end{biography}

%% sample for biography without author's image
%\begin{biography}{}{\author{Author Name.} This is sample author biography text this is sample author biography text this is sample author biography text this is sample author biography text this is sample author biography text this is sample author biography text this is sample author biography text this is sample author biography text.}
%\end{biography}

$for(include-after)$
$include-after$
$endfor$

\end{document}
